\urlsection{http://www2.cs.uh.edu/~gnawali/courses/cosc6377-s21/papers/KatzBassett10.pdf}{Reverse
Traceroute}
\begin{itemize}
    \item traceroute measures sequence of routers from source to destination,
    with RTT at each hop
    \item attempt to build a reverse path tool equivalent to traceroute
    \item source requests a path from system, coordinates probes from source and
    set of distributed vantage points
    \begin{itemize}
        \item issue traceroutes to source, yielding atlas of paths to it
        \item use this limited view to bootstrap measurement of desired path
    \end{itemize}
    \item once path from destination reaches hop in the atlas, use atlas to
    derive remainder of path
    \item identify reverse hops with IP options
    \begin{itemize}
        \item $\texttt{RR-Ping}(S\to D)$
        \begin{itemize}
            \item $S$ issues ICMP Echo Request to probe to $D$ with RR option
            \item if RR slots remain on response, routers record some of route
            \item allows limited measurement of reverse path as long as
            destination is fewer than 9 hops from $S$
        \end{itemize}
        \item $\texttt{TS-Query-Ping}(S\to D\mid D, R)$
        \begin{itemize}
            \item $S$ issues ICMP ping probe to $D$ with timestamp query
            \item $R$ records timestamp only if encountered by probe after $D$
            has stamped packet
            \item if $S$ receives timestamp for $R$, then knows $R$ appears on
            reverse path
        \end{itemize}
    \end{itemize}
    \item incrementally build paths
\end{itemize}